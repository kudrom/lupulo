%%%%%%%%%%%%%%%%%%%%%%%%%%%%%%%%%%%%%%%%%
% Beamer Presentation
% LaTeX Template
% Version 1.0 (10/11/12)
%
% This template has been downloaded from:
% http://www.LaTeXTemplates.com
%
% License:
% CC BY-NC-SA 3.0 (http://creativecommons.org/licenses/by-nc-sa/3.0/)
%
%%%%%%%%%%%%%%%%%%%%%%%%%%%%%%%%%%%%%%%%%

\documentclass{beamer}
\usepackage[utf8]{inputenc}
\beamertemplatenavigationsymbolsempty

\mode<presentation> {
    \usetheme{Singapore}
    \usecolortheme{rose}
}

\usepackage{graphicx}
\usepackage{booktabs}
\usepackage{hyperref}
\usepackage{color}

\title[Final report of lupulo]{Final report of the development of \textbf{lupulo}}

\author{Alejandro López Espinosa}
\institute[VIVES university]
{
    VIVES University
    \medskip
}
\date{\today}

\begin{document}
    \begin{frame}
    \titlepage
    \end{frame}

    \begin{frame}
        \frametitle{Table of contents}
        \tableofcontents
    \end{frame}

    \section{Overview}
    \begin{frame}
        \frametitle{Useful info}
        lupulo is developed as a free software project under the GPL
        license.

        lupulo's backend currently \textbf{is built with python2} because
        twisted is not written to be compatible with python3.

        lupulo is currently at the 0.3.0 stable release.
        \\~\\

        You can find the source code in 
        \textcolor{orange}{\href{http://github.com/kudrom/lupulo}{github.com/kudrom/lupulo}}

        You can download lupulo with pip.

        The docs are updated in ReadTheDocs.
    \end{frame}

    \begin{frame}
        \frametitle{Goals}

        \begin{block}{What's lupulo}
            lupulo is a web framework to build realtime web pages that monitor
            and/or command a device.
        \end{block}

        Only two goals:
        \begin{itemize}
            \item Easiness of development
            \item Extensibility of complex behaviour
        \end{itemize}

        In order to be able to provide those two goals, the framework provides
        two sets of abstractions, the ones targeted at \textbf{beginner} users
        and the ones targeted at \textbf{experienced} users.
    \end{frame}

    \begin{frame}
        \frametitle{Main abstractions}
        \begin{itemize}
            \item \color{blue} Data schema language
            \item Layout language
            \item Templates
            \item \color{purple} Widgets
            \item Accessors
            \item Listeners
        \end{itemize}

        All of this abstractions provide a smooth \textbf{workflow} that both
        type of users should follow to develop a web page.
    \end{frame}

    \section{Workflow}
    \begin{frame}
        \frametitle{Create a valid project}
        You can install the software with:
            
        \texttt{pip install lupulo}

        Once you have installed lupulo, you type \texttt{lupulo\_create} in a
        directory to create a valid lupulo project.

        It's going to create a bunch of directories and files that you're going
        to change later on.

        One of those files is the \textbf{settings.py} file which you'll modify
        to configure the backend mostly.
    \end{frame}

    \begin{frame}
        \frametitle{Write the data schema}

        The data schema is made of \textbf{source events} that relate to every
        sensory information the device is sending.

        Each source event is described by a set of obligatory parameters
        depending of its \textbf{type}.

        There are two kind of types, the \textbf{primitive} and the
        \textbf{aggregated}.

        The primitive type describes some raw data like \textbf{number}, 
        \textbf{enum} or \textbf{date}.

        The aggregated type is an aggregation of other types. There are two
        aggregated types: \textbf{dict} and \textbf{list}.

        \textbf{There is no limit to aggregation}.
    \end{frame}

    \begin{frame}
        \frametitle{Write the layout}
        A widget is some \textbf{JS object} that renders information in the web
        page.

        A layout is a \textbf{description} of a widget, it describes how the
        widget is going to behave in the web page.

        In  the layout the user \textbf{binds} a particular widget with a
        particular data source event.

        A layout can \textbf{inherit} from another layout some or all of its
        parameters.

        \textbf{There is no limit to the levels of inheritance.}

        Multiple inheritance is not allowed.
    \end{frame}

    \begin{frame}
        \frametitle{Templates}
        You can modify the html pages with \textbf{jinja2} templates.

        At the moment lupulo provides some base templates that you can extend.

        You can design your own url sitemap and use RESTful principles for the
        command interface. You only need to write your own \textbf{urls.py} file
        in the same fashion that django does.

        Each url must be handled by a \textbf{twisted Resource}.

        The templates are rendered \textbf{asynchronously}.
    \end{frame}

    \begin{frame}
        \frametitle{Launch}
        Once you have written everything, you launch the server with
        \texttt{lupulo\_start} in the 8080 port (by default).

        You can edit the layout and data schema \textbf{without the need to
        restart the web server}.

        You can use the \textbf{debug web page} to see how things are working in
        the frontend.

        You can use a \textbf{standalone sse client} to see what the backend is
        sending through the sse data connection.
    \end{frame}

    \begin{frame}
        \frametitle{Write widget}
        If you want to visualize some information in a new way, you can write
        your own widget and use them later in your layout file.

        You have to call the \textbf{Widget supertype} with the this object and
        the layout as parameters.

        The supertype will add some attributes to the object that will allow you
        later on to modify the web page.

        You can \textbf{aggregate} other types of widgets by constructing them
        in your constructor and calling them later to modify the web page.

        The framework is going to call \textbf{paint} each second and
        \textbf{clear\_framebuffers} when it needs to clear the visualization
        that the widget has created in the web page.

        You might need to use the \textbf{accessors} abstraction.

        You should provide \textbf{dynamic sizing}.
    \end{frame}

    \begin{frame}
        \frametitle{Accessors}
        The accessors abstraction allow a widget to access the data
        \textbf{without knowing its schema}.

        The widget \textbf{delegates} the real access of the data to another
        object that is constructed with the data schema and that retrieves the
        data the widget needs.

        The idea is to allow the programmer of the web page to describe in the
        layout the data it wants a widget to render \textbf{paying attention to
        the widget needs and the structure of the data}.

        Once the accessors are described in a layout, the widget will construct
        them with the \textbf{get\_accessors} function and will call them in the
        paint function to get the data it has to render.
    \end{frame}

    \begin{frame}
        \frametitle{Write listener}
        In order to connect the backend to the data source, the user can
        build its own listener that will retransmit to the backend the data
        it receives.
        
        A listener is a \textbf{twisted service} that will be run when the
        server is started.

        Once the listener is created by the backend, it will \textbf{publish}
        data it receives from the device to a twisted resource that will
        validate and push the information to the frontend.
    \end{frame}

    \section{Architecture}
    \begin{frame}
        \frametitle{Main components}
        There are two main components of the project, the \textbf{backend},
        which is built with python2 and twisted and the \textbf{frontend}
        which is built in javascript.

        These two main components communicate through an asynchronous data link
        provided by a HTML5 API called \textbf{Server Sent Events}, which
        provides a unidirectional stream to push information from the server to
        the browser.

        The user doesn't see this asynchronous data link, it only configures
        what widgets listen to what data sources.
    \end{frame}

    \begin{frame}
    \Huge{\centerline{Diagrams}}
    \end{frame}

    \section{Demo}
    \begin{frame}
    \Huge{\centerline{Demo}}
    \end{frame}

    \begin{frame}
    \Huge{\centerline{Thanks for your attention.}}
    \end{frame}

    \section{Work to do}
    \begin{frame}
        \frametitle{The future}
        For future releases it could be nice to:
        \begin{itemize}
            \item Expand the number of listeners
            \item Expand the number of widgets
            \item Review both high level languages
            \item More granularity in the paint loop
        \end{itemize}
    \end{frame}

\end{document} 
