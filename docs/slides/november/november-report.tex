%%%%%%%%%%%%%%%%%%%%%%%%%%%%%%%%%%%%%%%%%
% Beamer Presentation
% LaTeX Template
% Version 1.0 (10/11/12)
%
% This template has been downloaded from:
% http://www.LaTeXTemplates.com
%
% License:
% CC BY-NC-SA 3.0 (http://creativecommons.org/licenses/by-nc-sa/3.0/)
%
%%%%%%%%%%%%%%%%%%%%%%%%%%%%%%%%%%%%%%%%%

\documentclass{beamer}
\usepackage[utf8]{inputenc}
\beamertemplatenavigationsymbolsempty

\mode<presentation> {
    \usetheme{Singapore}
    \usecolortheme{rose}
}

\usepackage{graphicx}
\usepackage{booktabs}
\usepackage{hyperref}
\usepackage{color}

\title[Overview of lupulo]{Report of the progress made in \textbf{lupulo}}

\author{Alejandro López Espinosa}
\institute[VIVES university]
{
    VIVES University
    \medskip
}
\date{\today}

\begin{document}
    \begin{frame}
    \titlepage
    \end{frame}

    \begin{frame}
        \frametitle{Table of contents}
        \tableofcontents
    \end{frame}

    \section{Overview}
    \begin{frame}
        \frametitle{Useful info}
        lupulo is developed as a free software project under the GPL
        license (for the moment).

        lupulo's backend currently \textbf{is built with python2} because
        twisted is not written to be compatible with python3.

        lupulo is currently at the 0.1.0 stable release.
        \\~\\

        You can find the source code in 
        \textcolor{orange}{\href{http://github.com/kudrom/lupulo}{github.com/kudrom/lupulo}}

        You can download lupulo with pip.

        The docs are updated in ReadTheDocs.
    \end{frame}

    \begin{frame}
        \frametitle{Goals}
        \begin{itemize}
            \item Dynamic description of the data the device is sending
            \item Allow big modifications of the web page
            \item Multiple web pages for a single device
            \item Ease of development of the web pages
            \item Record the data for offline analysis
            \item Reuse of code
            \item Extensibility of the behaviour
        \end{itemize}
        \vspace{1cm}

        \begin{block}{What's lupulo}
            lupulo is a web framework to build realtime web pages that monitor
            and/or command the state of a device.
        \end{block}
    \end{frame}

    \begin{frame}
        \frametitle{Main abstractions}
        \begin{itemize}
            \item Data schema language
            \item Layout language
            \item Templates
            \item Widgets
            \item Accessors
            \item Listeners
        \end{itemize}
    \end{frame}

    \section{How to use it}
    \begin{frame}
        \frametitle{Create a valid project}
        You \textbf{can} install the software with:
            
        \texttt{pip install lupulo}

        But you \textbf{should} build your own distribution from the source 
        code and later install it from your filesystem with pip. You can use a 
        Makefile that does just that and something more.

        \vspace{0.5cm}

        Once you have installed lupulo, you type \texttt{lupulo\_create} in a
        directory to launch the server.

        It's going to create a bunch of directories and files that you're going
        to change later on.

        One of those files is the \textbf{settings.py} file which you'll modify
        to configure the backend mostly.
    \end{frame}

    \begin{frame}
        \frametitle{Write the data schema}

        The data schema is made of \textbf{source events} that relate to every
        sensory information the device is sending.

        Each source event is described by a set of obligatory parameters
        depending of its \textbf{type}.

        There are two kind of types, the \textbf{primitive} and the
        \textbf{aggregated}.

        The primitive type describes some raw data like \textbf{number}, 
        \textbf{enum} or \textbf{date}.

        The aggregated type is an aggregation of other types. There are two
        aggregated types: \textbf{dict} and \textbf{list}.

        \textbf{There is no limit to aggregation}.
    \end{frame}

    \begin{frame}
        \frametitle{Write the layout}
        A widget is some \textbf{JS object} that renders information in the web
        page.

        A layout is a \textbf{description} of a widget, it describes how the
        widget is going to behave in the web page.

        In  the layout the user \textbf{binds} a particular widget with a
        particular data source event.

        A layout can \textbf{inherit} from another layout some or all of its
        parameters.

        \textbf{There is no limit to the levels of inheritance.}

        Multiple inheritance is not allowed.
    \end{frame}

    \begin{frame}
        \frametitle{Templates}
        You can modify the html pages with \textbf{jinja2} templates.

        At the moment lupulo provides some base templates that you can extend.

        You can design your own url sitemap and use RESTful principles for the
        command interface. You only need to write your own \textbf{urls.py} file
        in the same fashion that django does.

        Each url must be handled by a \textbf{twisted Resource}.
    \end{frame}

    \begin{frame}
        \frametitle{Launch}
        Once you have written everything, you launch the server with
        \texttt{lupulo\_start}.

        You can test that everything is working as expected with a standalone
        sse client called \texttt{lupulo\_sse\_client} or you can go to
        localhost:8080 and enjoy your first lupulo realtime web page.
    \end{frame}

    \section{Architecture}
    \begin{frame}
        \frametitle{Main components}
        There are two main components of the project, the \textbf{backend},
        which is built with python2 and twisted and the \textbf{frontend}
        which is built in javascript.

        These two main components communicate through an asynchronous data link
        provided by a HTML5 API called \textbf{Server Sent Events}, which
        provides a unidirectional stream to push information from the server to
        the browser.

        The user doesn't see this asynchronous data link, it only configures
        what widgets listen to what data sources.
    \end{frame}

    \begin{frame}
        \frametitle{Backend}
        The backend is responsible of:

        \begin{itemize}
            \item Compile the data schema
            \item Compile the layout
            \item Implement a web server
            \item Push information to the frontend whenever it's available
            \item Provide the listeners abstraction
            \item Provide the templates abstraction
            \item Provide hot layout and data schema
        \end{itemize}
    \end{frame}

    \begin{frame}
        \frametitle{Frontend}
        The frontend is responsible of:

        \begin{itemize}
            \item Initiate the connection towards the server
            \item Render the information
            \item Provide the widgets abstraction
            \item Provide the accessors abstraction
        \end{itemize}
    \end{frame}

    \section{Future releases}
    \begin{frame}
        \frametitle{The future}
        I'm working currently on (for the 0.2.0 release):
        \begin{itemize}
            \item The templates abstraction
            \item Building a smart debug web page
            \item Building a smart web page for mongodb
            \item Expanding the number of default widgets
            \item Spotting and fixing bugs
        \end{itemize}

        I'd like to work on:
        \begin{itemize}
            \item Expanding the number of default listeners
            \item Test the framework with real devices
            \item Think about what more abstractions I can provide in the
                frontend and the embedded device
        \end{itemize}
    \end{frame}

    \section{Demo}
    \begin{frame}
    \Huge{\centerline{Demo}}
    \end{frame}

    \begin{frame}
    \Huge{\centerline{Any question?}}
    \end{frame}

\end{document} 
